\documentclass[11pt]{article}
\usepackage[utf8]{inputenc}
\usepackage{amsfonts}
\usepackage{natbib}
\usepackage{graphicx}
\usepackage{amsmath}
\usepackage{amssymb}
\usepackage{mathrsfs} % Cursive font
\usepackage{graphicx}
\usepackage{ragged2e}
\usepackage{fancyhdr}
\usepackage{nameref}
\usepackage{wrapfig}
\usepackage{hyperref}


\usepackage{mathtools}
\usepackage{xparse} \DeclarePairedDelimiterX{\Iintv}[1]{\llbracket}{\rrbracket}{\iintvargs{#1}}
\NewDocumentCommand{\iintvargs}{>{\SplitArgument{1}{,}}m}
{\iintvargsaux#1}
\NewDocumentCommand{\iintvargsaux}{mm} {#1\mkern1.5mu,\mkern1.5mu#2}

\makeatletter
\newcommand*{\currentname}{\@currentlabelname}
\makeatother



\addtolength{\textwidth}{0.2cm}
\setlength{\parskip}{8pt}
\setlength{\parindent}{0.5cm}
\linespread{1.5}

\pagestyle{fancy}
\fancyhf{}
\rhead{TP1 - Cipullo}
\lhead{Seguridad Informática}
\rfoot{\vspace{1cm} \thepage}

\renewcommand*\contentsname{\LARGE Índice}

\setlength{\skip\footins}{0.5cm}

\begin{document}

\begin{titlepage}
    \hspace{-2.5cm}\includegraphics[scale= 0.48]{../header.png}
    \begin{center}
        \vfill
            \noindent\textbf{\Huge Seguridad Informática}\par
            \vspace{.5cm}
            \noindent\textbf{\Huge Trabajo Práctico 1}\par
            \vspace{.5cm}
        \vfill
        \noindent \textbf{\huge Alumna:}\par
        \vspace{.5cm}
        \noindent \textbf{\Large Cipullo, Inés}\par
 
        \vfill
        % \large Universidad Nacional de Rosario \par
        \noindent\large 2023
    \end{center}
\end{titlepage}
\ \par

\section*{Ejercicio 3}
Las politicas de seguridad, y en particular su implementación, suelen agregarle restricciones al 
software. Esto significa, por lo general, que la experiencia del usuario se ve negativamente afectada, 
lo cual agrega fricción al proceso de venta. Además, se necesita cierto entrenamiento tanto del equipo 
comercial del producto como de los clientes sobre las politicas.

\section*{Ejercicio 9}
Por un lado, el obfuscamiento de código que se encuentra al acceso del público general, una técnica 
bastante conocida de ocultamiento de la información, se considera como que no agrega apreciable 
seguridad al sistema. Esto es principalmente porque no es usual que las vulnerabilidades se descrubran 
y exploten leyendo el código, pero también porque esto no cambia el significado del programa, y 
eventualmente se lograra deducir, aunque se tarde un poco más porque el código esta obfuscado. 

Por otro lado, una práctica de ocultamiento de la información que si agrega seguridad es encriptar los 
datos, ya sea porque están en tránsito o son archivos que contienen secretos. La encriptación es de 
gran importancia dentro de la protección de la información.
 

\section*{Ejercicio 13}
\subsection*{a.}
El modelo de Biba para integridad, denominado \textit{Biba Integrity Model} es un sistema formal de 
estados/transaciones sobre politicas de seguridad computacional que describe un conjunto de reglas 
de control de acceso diseñadas para asegurar la integridad de los datos (correctitud y completitud 
de los datos). Los datos y los usuarios son agrupados en niveles de integridad ordenados. El modelo 
está armado como para que los sujetos no puedan atentar contra la integridad de datos en un nivel 
mayor al del sujeto, o que la integridad del sujeto no se vea comprometida por datos de un nivel 
menor que el sujeto.

La idea del modelo es evitar que los datos fluyan de niveles bajos hacia niveles altos, para que los
datos de alto nivel no modifiquen su integridad basandose en datos de bajo nivel. 

\subsection*{b.}
Se podrían implementar los modelos de Bell-LaPadula y Biba en simultaneo para modelar confidencialidad 
e integridad a la par, suponiendo la misma cantidad de niveles en ambos. Depende de como se decidan 
mapear los niveles de seguridad, el sistema permitirá mayor o menor circulación de datos. 

Si se mapea el mayor nivel de confidencialidad con el mayor nivel de integridad (y sucesivamente en 
los menores niveles), resulta en un modelo donde la única circulación de datos habilitada es entre 
objetos del mismo nivel. No podrá haber circulación de datos niveles distintos porque BLP prohíbe un 
sentido, y Biba el otro. En cambio, si se mapea el mayor nivel de confidencialidad con el menor nivel 
de integridad (y el menor nivel de confidencialidad con el mayor de integridad, y así sucesivamente), 
resulta en un modelo con circulación de datos entre objetos del mismo nivel y también desde objetos 
de menor nivel de confidencialidad hacia objetos de mayor nivel de confidencialidad, unidireccionalmente. 
Esto es porque ese sentido de circulación de los datos está permitido por ambos modelos, según como se 
acomodaron los niveles.

\section*{Ejercicio 21}
Tenemos el siguiente programa:

\begin{verbatim}
-- Prog 3
x = readH();
writeL("Loading");
while (x-- > 100)
    writeL(".");
writeH(x);
\end{verbatim}

Notamos que hay flujo indebido de información de H a L, esto es porque la guarda del bucle \verb|while| 
depende de la variable \verb|x|, que es de nivel H, pero luego dentro del bucle se escribe sobre 
variables L. Más explicitamente, la condición del bucle va decreciendo el valor de \verb|x| en cada 
iteración y luego dentro del bucle se imprime un punto \verb|"."| en un canal de nivel L. Esto 
significa que si hay un proceso L del otro lado del canal leyendo todos los puntos que se escriben, 
podrá saber el valor de \verb|x|, que es de nivel H y cuyo valor es potencialmente un secreto.

\section*{Ejercicio 22}
Un Sistema Operativo que implementa el mecanismo de \textit{multi-ejecución segura} (SME) basicamente 
hará un \verb|fork| de un proceso cada vez que accede a datos con una clase de acceso más alta que la 
del proceso actual, y le asignará al nuevo proceso la clase de acceso de los datos de mayor nivel. 

En el ejemplo presentado en el Ejercicio 21, una ejecución bajo SME suponiendo que hay dos niveles 
de acceso (L y H) funcionaría como sigue:

\begin{itemize}
    \item Apenas se lanza el proceso es clasificado como L.
    \item Al intentar asignarle a \verb|x| un valor H, el OS hace un \verb|fork| y crea un nuevo proceso a nivel H, resultando en dos procesos, $P_L$ (que ya existía) y uno nuevo, $P_H$.
    \item $P_L$ no tiene permiso para acceder a información H, por lo que el OS va a asignarle un valor constante a la variable \verb|x|, dependiendo de como esté implementado. Luego se ejecutará el bucle sin restricciones, dependiendo del valor de \verb|x|, que podiblemente sea nulo. Finalmente, se esribirá el valor resultante en \verb|x| en un canal H, para lo cual $P_L$ tiene permisos.
    \item Por su lado, en $P_H$ la variable \verb|x| podrá tomar su valor real ya que habrá permisos para leer información H y se ejecutará el bucle pero sin poder escribir en L, ya que no contará permisos para eso. Finalmente se escribirá el nuevo valor de \verb|x| en H.
\end{itemize}

En resumen, va a haber dos procesos desde el inicio del programa, $P_L$ y $P_H$ con permisos L y H 
respectivamente. $P_L$ ejecutará todo el programa pero sin el valor real de \verb|x|, mientras que 
$P_H$ si tendrá el valor real de \verb|x| pero no podrá ejecutar las sentencias de escritura en L.

\section*{Ejercicio 23}
El paso 5 del desafío \textit{Information Flow Challenge} (que se puede encontrar  
\href{https://ifc-challenge.appspot.com/steps/allergy}{acá}) lo resolví de la siguiente manera:

\begin{verbatim}
try {
    l = true;
    if (h) {throw;} else {skip;}
    l = false;
} catch {
    skip;
}
\end{verbatim}

Se logra realizar un leak de información de \verb|h| a \verb|l|. Esto es porque si bien se categoriza 
el \verb|throw| como H por estar dentro de un \verb|if| con guarda H, y entonces no se podría 
asignarle valor a una variable L en el \verb|catch|, basta con asignarle el valor a \verb|l| fuera 
del \verb|if| y luego dependiendo del valor de \verb|h|, tirar la excepción o no. El \verb|throw| 
entonces funciona como un GOTO, permitiendo saltear cierta parte del código, cambiar de contexto. De 
esta manera, se mantiene el valor previamente asignado a \verb|l| o se modifica, de acuerdo al valor 
de \verb|h|.

Un cambio que se le podría hacer a las reglas de tipado para evitar un leak de información como el 
antes descripto, es no permitir escritura en variables L dentro del \verb|try| luego de una sentencia 
H. De esa manera se aseguraría que el \verb|throw| no se utilice para cambiar de contexto dependiendo 
de una variable H, ya que dejaría de haber dos contextos L luego del \verb|throw|.

\end{document}